%%%%%%%%%%%%%%%%%%%%%%%%%%%%%%%%%%%%%%%%%%%%%%%%%%%%%%%%%%%%%%%%%%%%%%%%%%%%%%%%
%%%%%%%%%%%%%%%%%%%%%%%%%%%%%%%%%%%%%%%%%%%%%%%%%%%%%%%%%%%%%%%%%%%%%%%%%%%%%%%%
%%                          AUTHOR: BIBEKANANDA DATTA                         %%
%%                             (C) SEPTEMBER 2023                             %%
%%                      PhD STUDENT, MECHANICAL ENGINEERING                   %%
%%                           JOHNS HOPKINS UNIVERSITY                         %%
%%%%%%%%%%%%%%%%%%%%%%%%%%%%%%%%%%%%%%%%%%%%%%%%%%%%%%%%%%%%%%%%%%%%%%%%%%%%%%%%
%%%%%%%%%%%%%%%%%%%%%%%%%%%%%%%%%%%%%%%%%%%%%%%%%%%%%%%%%%%%%%%%%%%%%%%%%%%%%%%%
%%             PLEASE CHECK THE README.md FILE BEFORE YOU PROCEED             %%
%%              it may be convenient to read this file on GitHub              %%
% https://github.com/bibekananda-datta/JH-MechE-Dissertation-Proposal-Template %
%% template hosted on the GitHub repository is likely to be the most updated  %%
%%%%%%%%%%%%%%%%%%%%%%%%%%%%%%%%%%%%%%%%%%%%%%%%%%%%%%%%%%%%%%%%%%%%%%%%%%%%%%%%


%%%%%%%%%%%%%%%%%%%%%%%%%%%%%%%%%%%%%%%%%%%%%%%%%%%%%%%%%%%%%%%%%%%%%%%%%%%%%%%%
% this is an unofficial template for the thesis or dissertation proposal in 
% the Department of Mechanical Engineering at Johns Hopkins University. 
% consult with the department or advisor before using the template.
% this template includes comments on the department-suggested page limits. 
% however, it is the user's responsibility to ensure the formatting conforms 
% with the advisor or proposal committee or department's requirement.
% this template is based on the LaTeX article class and uses BibLaTeX as 
% bibliographic manager. Use a citation manager to generate the BibLaTeX file.
%%%%%%%%%%%%%%%%%%%%%%%%%%%%%%%%%%%%%%%%%%%%%%%%%%%%%%%%%%%%%%%%%%%%%%%%%%%%%%%%



%%%%%%%%%%%%%%%%%%%%%%%%%%%%%%%%%%%%%%%%%%%%%%%%%%%%%%%%%%%%%%%%%%%%%%%%%%%%%%%%
%% if possible, please make your formatting changes here through the variables 
%% go through all the variables and understand what role they play in formatting
%%%%%%%%%%%%%%%%%%%%%% LIST OF VARIABLES FOR FORMATTING %%%%%%%%%%%%%%%%%%%%%%%%

\def\FontPackage{lmodern}                   % latin modern font (you can also change it to times)
\def\BibFileName{references.bib}            % name of BibLaTeX file with all the bibliography

\def\NoSectionLevel{3}                      % 3 levels for sections ... to subsubsection
\def\TocIndent{0}                           % indentation in the list of figs and tables
\def\NoTocLevel{3}                          % no of levels showed in the table of contents
%% 3 levels mean section to subsubsection.. decrease if you want less to show in TOC

\def\GlobalMargin{1.0in}                    % margin on all sides

%%%% font size and typeset for different environments
%% check here for details: https://en.wikibooks.org/wiki/LaTeX/Fonts
\def\BaseFont{12pt}
\def\TitleFont{\Large\bfseries\MakeUppercase}
\def\SectionFont{\large\bfseries}           % section heading font format
\def\SubsectionFont{\normalsize\bfseries}   % subsection heading font format
\def\SubsubsectionFont{\normalsize\itshape} % subsubsection heading font format
\def\CaptionFontSize{small}                 % caption font size
\def\CaptionFontType{bf}                    % boldface label for captions
\def\CaptionSeparator{colon}                % separates caption heading from text. can use 'period' as well

\def\TitlePageSpacing{\singlespacing}       % spacing of the title page contents
\def\MainTextSpacing{\singlespacing}        % double spacing in the main text (JH library requirement)
\def\TOCTextSpacing{\onehalfspacing}        % one-half spacing for TOC texts
\def\ParagraphSpacing{\baselineskip}        % spacing between paragraph
\def\ParagraphIndent{0}                     % indentation at the beginning of the paragraph
\def\FullCiteSpacing{1.0}                   % spacing in a fullcite item
\def\LOTItemSpacing{3 pt}                   % spacing between LOT/LOF items
\def\BibItemSpacing{0.4\baselineskip}       % spacing between bibliographic items in reference
\def\FootnoteSpacing{0.5\baselineskip}      % spacing between footnotes
\def\CaptionSpacing{0}                      % spacing between the figure and the caption (unit: pt)

\def\GlobalTableSpacing{1.0}                % global spacing parameter for table
%% if this seems too widespread for you, try changing it locally using
%% \begin{group} ... \renewcommand{\arraystretch} ... \end{group} commands

%%%%%%%%%%%%%%%%%%%% END LIST OF VARIABLES FOR FORMATTING %%%%%%%%%%%%%%%%%%%%%%




%%%%%%%%%%%%%%%%%%%%%%%%%%%%%%%%%%%%%%%%%%%%%%%%%%%%%%%%%%%%%%%%%%%%%%%%%%%%%%%%
%% add packages as needed but sometimes the order of the packages matters.
%% I primarily tried to load the packages in alphabetical order or based on
%% their functionalities (all math/ table packages) unless there is an issue
%% with package dependency, so I included packages in that order. remember,
%% you may get warnings/ errors for the order in which packages are included
%% you may have to change the options in biblatex package for the bibliography
%%%%%%%%%%%%%%%%%%%%%%%%%%% LaTeX CLASS AND PACKAGES %%%%%%%%%%%%%%%%%%%%%%%%%%%

\documentclass[\BaseFont,letterpaper]{article} % document class (article w/ 12 pt)

\usepackage[utf8]{inputenc}	                % for encoding input character
\usepackage[pagewise,mathlines]{lineno}     % linenumbers


%% math packages
\usepackage{amsfonts,amssymb,amsmath,amsthm,autobreak,cancel,dsfont,mathtools,mathbbol,mathrsfs,siunitx,upgreek}

\usepackage[ruled]{algorithm2e}             % to manage algorithm environment
\usepackage[titletoc]{appendix}             % to manage appendix chapters
\usepackage[american]{babel}                % for different language typography

%% bibliographic package (make sure your bib file is in BibLaTeX format)
%% use Zotero or some other reference manager to generate the BibLaTeX file
%% change the style or other options if you need to
\usepackage[backend=biber, defernumbers=true, style=nature, maxnames=9, date=year, isbn=false, url=false, doi=true]{biblatex}
% \usepackage[backend=biber, defernumbers=true, style=apa, isbn=false, url=false, doi=true]{biblatex}

\usepackage{blindtext}                      % to generate random filler texts
\usepackage{calc}                           % to set arithmetic arguments for spacing
\usepackage{caption}                        % to manage captions
\usepackage{color}                          % color related packages
\usepackage{csquotes,epigraph,varwidth}     % for managing quotes
\usepackage{enumitem}                       % to manage list environment
\usepackage{float}                          % to manage floating environment
\usepackage[T1]{fontenc}                    % for font encoding
\usepackage[bottom]{footmisc}               % footnote environment management
\usepackage{graphicx,wrapfig}               % to manage images
\usepackage{geometry}                       % to manage margins and others
\usepackage{fancyhdr}                       % for header/ footer settings
\usepackage[dvipsnames]{xcolor}
\usepackage[a-1b]{pdfx}                     % to generate PDF/A file (before hyperref)
\usepackage[pdfa]{hyperref}                 % for hyperlinks
\usepackage[all]{hypcap}                    % for captions on the side of figures
\usepackage{ifthen}                         % if-then statement in algorithm
\usepackage{lscape}                         % landscape mode
\usepackage{listings}                       % to include codes

%% table related packages
\usepackage{booktabs,longtable,makecell,multicol,multirow,tabularx,xltabular}

\usepackage{tocloft}                        % to manage table of contents
\usepackage{parskip}                        % paragraph spacing 
\usepackage{setspace}                       % sets space between lines
\usepackage{seqsplit}                       % splits long character sequence
\usepackage[rightcaption]{sidecap}          % for sideway captions
\usepackage{titlesec}                       % managing different titles
\usepackage[absolute]{textpos}              % to position text
\usepackage{tikz}                           % package for drawing
\usepackage{subcaption}                     % individual panel and caption

%% add more packages and options as you need

%%%%%%%%%%%%%%%%%%%%%%%%% END LaTeX CLASS AND PACKAGES %%%%%%%%%%%%%%%%%%%%%%%%%



%%%%%%%%%%%%%%%%%%%%%%%%%%%%% DOCUMENT FORMATTING %%%%%%%%%%%%%%%%%%%%%%%%%%%%%%

%%% choice a font form (or add something else) for your thesis (uncomment one option)
\usepackage{\FontPackage}       
% if you want to use Palatino font, use the following and comment above line
% \usepackage[sc]{mathpazo}                  % palatino font family

%%% I use Zotero to generate the BibLaTeX file and include it in the same directory
\addbibresource{\BibFileName}

%%% margin settings with geometry package
\geometry{margin=\GlobalMargin, nomarginpar}

%%% settings for the hyperref package
\hypersetup{linktocpage, unicode, linktoc=all, colorlinks=true, citecolor=blue, filecolor=blue, linkcolor=blue, urlcolor=blue}
\urlstyle{rm}   % to remove the default URL style (tt format)

%%% settings for figure caption
\captionsetup{belowskip=\CaptionSpacing pt, font=\CaptionFontSize, labelfont=\CaptionFontType, labelsep=\CaptionSeparator, hypcap=true} 

\setcounter{tocdepth}{\NoTocLevel}                      % list depth in ToC

\renewcommand{\cfttoctitlefont}{\SectionFont}           % font for ToC title
\renewcommand{\cftloftitlefont}{\SectionFont}           % font for LoF title
\renewcommand{\cftlottitlefont}{\SectionFont}           % font for LoT title

\setlength{\cftbeforesecskip}{0.5\baselineskip}         % space between sections in TOC
\renewcommand{\cftsecleader}{\cftdotfill{\cftdotsep}}   % dots for sections too

% tweak to LoF/ LoT to add 'Figure' & 'Table' to the figure and table caption listing
% to change the distance to the start of the figure/ table title
\setlength{\cftfigindent}{\TocIndent pt}                % indentation from figures in LoF
\renewcommand{\cftfigpresnum}{\bfseries Figure }
\setlength{\cftfignumwidth}{\widthof{\textbf{Figure~99.999~}}}
\setlength{\cftbeforetabskip}{\LOTItemSpacing}          % spacing between each item
%
\setlength{\cfttabindent}{\TocIndent pt}                % indentation from tables in LoT
\renewcommand{\cfttabpresnum}{\bfseries Table }
\setlength{\cfttabnumwidth}{\widthof{\textbf{Table~99.100~}}}
\setlength{\cftbeforefigskip}{\LOTItemSpacing}          % spacing between each item


%%%% font and style for the section, subsection, subsubsection, etc.
\setcounter{secnumdepth}{\NoSectionLevel}   % section to ... subsubsection ...
\titleformat{\section}{\SectionFont}{\thesection}{1em}{}[{\titlerule}]
\titleformat*{\subsection}{\SubsectionFont}
\titleformat*{\subsubsection}{\SubsubsectionFont}

%%%% paragraph, footnote, bib item, and table spacing
\setlength{\parskip}{\ParagraphSpacing}     % paragraph spacing
\setlength{\parindent}{\ParagraphIndent pt} % paragraph indentation
\setlength{\footnotesep}{\FootnoteSpacing}  % footnote separation
\setlength{\bibitemsep}{\BibItemSpacing}    % bib item separation 
\def\arraystretch{\GlobalTableSpacing}      % spacing in table

%%%% settings for math environment
\allowdisplaybreaks[1]                      % page break in math formula
\numberwithin{equation}{section}            % eqn no begins with section no
\setcounter{MaxMatrixCols}{20}              % maximum columns in matrix = 20


%%% settings for bibliography
\AtBeginBibliography{\urlstyle{rm}}
\DeclareFieldFormat{titlecase}{\MakeSentenceCase*{#1}}

\DeclareBibliographyCategory{mypapers}
\newcommand{\mybibexclude}[1]{\addtocategory{mypapers}{#1}}

%% settings for TikZ library (add more if you need them)
\usetikzlibrary{decorations.pathreplacing, positioning, arrows.meta, shapes,}


%%%%%%%%%%%%%%%%%%%%%%%%%%% END DOCUMENT FORMATTING %%%%%%%%%%%%%%%%%%%%%%%%%%%




%%%%%%%%%%%%%%%%%%%%%%%%%%%%%%%%%%%%%%%%%%%%%%%%%%%%%%%%%%%%%%%%%%%%%%%%%%%%%%%
%% add all your custom math settings and macros in the following section.
%% this is where LaTeX supremacy becomes a thing. you can customize a lot.
%%%%%%%%%%%%%%%%%%%%%%%%%%%%%% MATH MACROS %%%%%%%%%%%%%%%%%%%%%%%%%%%%%%%%%%%%

%%% Define math symbols and macros
\newcommand{\dC}{$^{\circ}$C}           % degree celsius symbol
\newcommand{\vect}[1]{\mathbf{#1}}      % boldface for vectors and tensors
\DeclareMathOperator{\T}{{\top}}        % transpose of a matrix/ tensor
\DeclareMathOperator{\tr}{tr}           % trace of a matrix
\DeclareMathOperator{\divg}{div}        % divergence of vector and tensor
\DeclareMathOperator{\grad}{grad}       % gradient of vector and tensor
\DeclareMathOperator{\curl}{curl}       % curl of vector and tensor

%% these are just some examples; add more macros for your custom symbols

%%%%%%%%%%%%%%%%%%%%%%%%%%%%% END MATH MACROS %%%%%%%%%%%%%%%%%%%%%%%%%%%%%%%%%


%%%%%%%%%%%%%%%%%%%%%%%%%%%%%%%%%%%%%%%%%%%%%%%%%%%%%%%%%%%%%%%%%%%%%%%%%%%%%%%
%% add all your non-mathematical macros and other random settings here.
%%%%%%%%%%%%%%%%%%%%%%%%%%%%%% OTHER MACROS %%%%%%%%%%%%%%%%%%%%%%%%%%%%%%%%%%%

\newcommand{\COMMENT}{\textcolor{red}}
\newcommand{\ADDCITATION}{\COMMENT{(ADD CITATION)}}

%% you can also add more simple comments here as you need
%% you can use some other packages for more complicated review and comment section

%%% to add small quotes
\newcommand{\say}[2]{\hfill\small\enquote{\textit{#1}}{ - \small\textsc{#2}.}}

%%%%%%%%%%%%%%%%%%%%%%%%%%%% END OTHER MACROS %%%%%%%%%%%%%%%%%%%%%%%%%%%%%%%%%
%%%%%%%%%%%%%%%%%%%%%%%%%%%%%%%%%%%%%%%%%%%%%%%%%%%%%%%%%%%%%%%%%%%%%%%%%%%%%%%

% \usepackage[color=red,unit=in,type=upperleft,showframe]{fgruler}

%%%%%%%%%%%%%%%%%%%%%%%%%%%%%%%%%%%%%%%%%%%%%%%%%%%%%%%%%%%%%%%%%%%%%%%%
%%%%%%%%%%%%%%%%%%%%%%%%%%%% BEGIN DOCUMENT %%%%%%%%%%%%%%%%%%%%%%%%%%%%

\begin{document}

% \linenumbers                      % may find it useful during drafting
%% place the command wherever you want to start numbering the line



%%%%%%%%%%%%%%%%%%%%%%%%%%%%%%%%%%%%%%%%%%%%%%%%%%%%%%%%%%%%%%%%%%%%%%%%
%%%%%%%%%%%%%%%%%%%%%%%%%%% BEGIN TITLE PAGE %%%%%%%%%%%%%%%%%%%%%%%%%%%

%%%%%%%%%%%%%%%%%%%%%%%%%%% TITLE AND AUTHOR %%%%%%%%%%%%%%%%%%%%%%%%%%%
\TitlePageSpacing \thispagestyle{empty}

\begin{center}
    Doctoral dissertation proposal submitted to the \\
    Department of Mechanical Engineering of The Johns Hopkins University
    
    \vspace{0.5in}                      % space between statement and title
    {\TitleFont {\LaTeX\ template for MechE dissertation proposal} \par}  
    %% tentative thesis title (\par is needed for proper spacing)
    
    \vspace{0.25in}                     % space between title and author
    
    Author Name                         % author name (student)
\end{center}
%%%%%%%%%%%%%%%%%%%%%%%%% END TITLE AND AUTHOR %%%%%%%%%%%%%%%%%%%%%%%%%


%%%%%%%%%%%%%%%%%%%%%%%%%%%% COMMITTEE MEMBERS %%%%%%%%%%%%%%%%%%%%%%%%%
\vspace{0.5in}
%%%% if you have more people on your committee then squeeze out some spaces or use the minipage feature
%%%% you can also consider shortening the affiliation to make more space to list more people
\textbf{Primary advisor:}

Dr. Chuck Darwin \\
Professor, \\
Department of Mechanical Engineering \\
Johns Hopkins University, Baltimore, MD


\textbf{Dissertation proposal committee members:} 

Dr. Albrecht Einstein \\
Professor, \\
Department of Mechanical Engineering \\
Johns Hopkins University, Baltimore, MD

Dr. Stewart Hawking \\
Professor, \\
Department of Mechanical Engineering \\
Johns Hopkins University, Baltimore, MD 


%%%%%%%%%%%%%%%%%%%%%%%%%% END COMMITTEE MEMBERS %%%%%%%%%%%%%%%%%%%%%%%

%%%%%%%%%%%%%%%%%%%%%%%%%%% TIME AND LOCATION %%%%%%%%%%%%%%%%%%%%%%%%%%
\vspace{0.5in}
\begin{center}
    Baltimore, Maryland \\          % location (according to Hopkins it's always the same)
    Month YEAR                      % like May 2024
    
    %% copyright statement is optional (but this will protect your ideas)
    {\begin{textblock*}{\textwidth}(\GlobalMargin,9.5in)
        \copyright\ YEAR Author Name. All rights reserved.
    \end{textblock*}
    \null}
\end{center}

%%%%%%%%%%%%%%%%%%%%%%%%% END TIME AND LOCATION %%%%%%%%%%%%%%%%%%%%%%%%

%%%%%%%%%%%%%%%%%%%%%%%%%%%% END TITLE PAGE %%%%%%%%%%%%%%%%%%%%%%%%%%%%
%%%%%%%%%%%%%%%%%%%%%%%%%%%%%%%%%%%%%%%%%%%%%%%%%%%%%%%%%%%%%%%%%%%%%%%%



%%%%%%%%%%%%%%%%%%%%%%%%%%%%%%%%%%%%%%%%%%%%%%%%%%%%%%%%%%%%%%%%%%%%%%%%
%%%%%%%%%%%%%%%%%%%%%%%%%%%%% FRONT MATTER %%%%%%%%%%%%%%%%%%%%%%%%%%%%%
%%%%%%%%%%%%%%%%%%%%%%%%%%%%% BEGIN ABSTRACT %%%%%%%%%%%%%%%%%%%%%%%%%%%

\clearpage 
\pagenumbering{roman}
\setcounter{page}{2}
\MainTextSpacing
\addcontentsline{toc}{section}{Abstract}
\section*{Abstract}

%%% write your abstract and keywords here
\blindtext

%%%%%%%%%%%%%%%%%%%%%%%%%%%%%% END ABSTRACT %%%%%%%%%%%%%%%%%%%%%%%%%%%%

%%%%%%%%%%%%%%%%%%%%%%%%%%%%%%%%% LISTS %%%%%%%%%%%%%%%%%%%%%%%%%%%%%%%%

%% single spacing appears to be too cramped and double spacing is too relaxed
{
\clearpage
\TOCTextSpacing                               % one-half spacing for the TOC
\hypersetup{linkcolor=black}                  % local hyperref settings to make the TOC appear black (you can comment it if you would like the default)
\renewcommand{\contentsname}{Table of Contents \vspace{3pt} \hrule}
\tableofcontents


% \MainTextSpacing
\clearpage \phantomsection
\addcontentsline{toc}{section}{List of Tables}
\renewcommand{\listtablename}{List of Tables \vspace{3pt} \hrule}
\listoftables


\clearpage \phantomsection
\addcontentsline{toc}{section}{List of Figures}
\renewcommand{\listfigurename}{List of Figures \vspace{3pt} \hrule}
\listoffigures
}

%%%%%%%%%%%%%%%%%%%%%%%%%%%%%%% END LISTS %%%%%%%%%%%%%%%%%%%%%%%%%%%%%%
%%%%%%%%%%%%%%%%%%%%%%%%%%% END FRONT MATTER %%%%%%%%%%%%%%%%%%%%%%%%%%%
%%%%%%%%%%%%%%%%%%%%%%%%%%%%%%%%%%%%%%%%%%%%%%%%%%%%%%%%%%%%%%%%%%%%%%%%


%%%%%%%%%%%%%%%%%%%%%%%%%%%%%%%%%%%%%%%%%%%%%%%%%%%%%%%%%%%%%%%%%%%%%%%%
%%%%%%%%%%%%%%%%%%%%%%%%%%% BEGIN MAIN TEXT %%%%%%%%%%%%%%%%%%%%%%%%%%%%

\clearpage \MainTextSpacing \pagenumbering{arabic}

%%%%%%%%%%%%%%%%%%%%%%%%%%%%%% BACKGROUND %%%%%%%%%%%%%%%%%%%%%%%%%%%%%%

\section{Background and significance}           %%% 2 page limit 

\blindtext \cite{dirac}.

\begin{figure}[ht]
\begin{center}
    \includegraphics[width=\textwidth, trim={6cm 5cm 6cm 5cm},clip,page=1] {figures.pdf}
    \caption{Here are some photos of ducks to make you feel happy in tough times.}
    \label{fig:ducks}
\end{center}
\end{figure}


%%%%%%%%%%%%%%%%%%%%%%%%%%%%%%%%%%%%%%%%%%%%%%%%%%%%%%%%%%%%%%%%%%%%%%%%%
%% for subsections and subsubsection, I didn't particularly like having 
%% numbered environments because they appeared confusing with my objectives  
%% and task numbering. so I used unnumbered subsections and subsubsections 
%% and added these env. to the table of contents using \phantomsection and 
%% \addcontentsline commands before and after the corresponding environment.
%%%%%%%%%%%%%%%%%%%%%%%%% RESEARCH OBJECTIVES %%%%%%%%%%%%%%%%%%%%%%%%%%%

\section{Research objectives}                   %%% suggested: 1 page limit

\blindtext \cite{knuthwebsite}

%% 
\phantomsection
\subsection*{Objective 1: Some major objective here}
\addcontentsline{toc}{subsection}{Objective 1: Some major objective here}

\blindtext

%% add more objectives (one objective is not enough you know it)





%%%%%%%%%%%%%%%%%%%%%%%%%%%%%%%%%%%%%%%%%%%%%%%%%%%%%%%%%%%%%%%%%%%%%%%%%
%% I used unnumbered subsections and subsubsections in this section as well
%% and added these env. to the table of contents using \phantomsection and 
%% \addcontentsline commands before and after the corresponding environment.
%%%%%%%%%%%%%%%%%%%%%%% METHODOLOGY AND RESULTS %%%%%%%%%%%%%%%%%%%%%%%%%

\section{Proposed methodology and results}      %%% suggested: 4 page limit

\blindtext

\subsection{Task 1: Major project task heading here related to objective 1}

\blindtext


\phantomsection
\subsubsection*{Subtask 1.1: Some sub-task here}
\addcontentsline{toc}{subsubsection}{Subtask 1.1: Some sub-task here}

\blindtext

\begin{table}[ht]
\centering
\begin{tabular}{c c c c} 
\toprule \toprule
Col1 & Col2 & Col2 & Col3 \\ 
\toprule \toprule
1 & 6 & 87837 & 787 \\ 
2 & 7 & 78 & 5415 \\
3 & 545 & 778 & 7507 \\
4 & 545 & 18744 & 7560 \\
5 & 88 & 788 & 6344 \\ 
\bottomrule
\end{tabular}
\caption{Table to test captions and labels taken from Overleaf.}
\label{table:1}
\end{table}


\subsubsection*{Subtask 1.2: Some other sub-task here}
\addcontentsline{toc}{subsubsection}{Subtask 1.2: Some other sub-task here}

\blindtext

%%% add more tasks and subtasks (one task is not enough for PhD - you know it by now)


%%%%%%%%%%%%%%%%%%%%%%%%%%%%%% PUBLICATIONS %%%%%%%%%%%%%%%%%%%%%%%%%%%%

\section{Planned publications}

%% add more items (if you have published/ planned for more papers)
%% \mybibexclude makes sure these papers do not appear in the bibliographic references.
%% in case you have the same paper somewhere else in the text and want it to appear 
%% as citations, remove the \mybibexclude{} command 

\begin{enumerate} [leftmargin=0.6cm,itemsep=-6pt]
    \item \fullcite{einstein}. \mybibexclude{einstein}
    \item \fullcite{knuth-fa}. \mybibexclude{knuth-fa} \hfill (In preparation)
\end{enumerate}



%%%%%%%%%%%%%%%%%%%%%%%%%%%%%%% TIMELINE %%%%%%%%%%%%%%%%%%%%%%%%%%%%%%

\section{Timeline}

%% this is an example of how to draw something using TikZ
%% I used a Timeline chart made using Excel/ PowerPoint combination
\begingroup
\begin{tikzpicture}
    % draw a horizontal line   
    \draw[thick, -Triangle] (0,0) -- (\textwidth,0) node[font=\scriptsize,below left=3pt and -8pt]{years};
    
    % draw vertical lines
    \foreach \x in {0,1,...,10}
    \draw (\x cm,3pt) -- (\x cm,-3pt);
    
    \foreach \x/\descr in {4/t-2,5/t-1,6/t,7/t+1}
    \node[font=\scriptsize, text height=1.75ex,
    text depth=.5ex] at (\x,-.3) {$\descr$};
    
    % colored bar up
    \foreach \x/\perccol in
    {1/100,2/75,3/25,4/0}
    \draw[lightgray!\perccol!red, line width=4pt] 
    (\x,.5) -- +(1,0);
    \draw[-Triangle, dashed, red] (5,.5) --  +(1,0);
    
    % colored bar down
    \foreach \x/\perccol in
    {3/100,4/75,5/0}
    \draw[lightgray!\perccol!green, line width=4pt] 
    (\x,-.7) -- +(1,0);
    \draw[-Triangle, dashed, green] (6,-.7) --  +(1,0);
    
    % braces
    \draw [thick ,decorate,decoration={brace,amplitude=5pt}] (4,0.7)  -- +(2,0) 
           node [black,midway,above=4pt, font=\scriptsize] {Training period};
    \draw [thick,decorate,decoration={brace,amplitude=5pt}] (6,-.9) -- +(-1,0)
           node [black,midway,font=\scriptsize, below=4pt] {Testing period};
\end{tikzpicture}
\captionof{figure}{An abstract timeline to finish my PhD. Drawing credit goes to a user on StackExchange.}
\endgroup

%%%%%%%%%%%%%%%%%%%%%%%%%%% ACKNOWLEDGEMENT %%%%%%%%%%%%%%%%%%%%%%%%%%%%

%% Optional acknowledgment section (uncomment if you want to use)
\section*{Acknowledgement}
\addcontentsline{toc}{section}{Acknowledgement}

%% acknowledge your mentor, collaborators, and labmates if they helped you craft your proposal or primary studies.
\blindtext

%%%%%%%%%%%%%%%%%%%%%%%%%%%%% BIBLIOGRAPHY %%%%%%%%%%%%%%%%%%%%%%%%%%%%%

\clearpage \phantomsection
\addcontentsline{toc}{section}{Bibliographic references}
\section*{Bibliographic references}

\printbibliography[heading=none,notcategory=mypapers]


%%%%%%%%%%%%%%%%%%%%%%%%%%% END BIBLIOGRAPHY %%%%%%%%%%%%%%%%%%%%%%%%%%%


%%%%%%%%%%%%%%%%%%%%%%%%%%%%% END MAIN TEXT %%%%%%%%%%%%%%%%%%%%%%%%%%%%
%%%%%%%%%%%%%%%%%%%%%%%%%%%%%%%%%%%%%%%%%%%%%%%%%%%%%%%%%%%%%%%%%%%%%%%%

\end{document}

%%%%%%%%%%%%%%%%%%%%%%%%%%%%% END DOCUMENT %%%%%%%%%%%%%%%%%%%%%%%%%%%%%
%%%%%%%%%%%%%%%%%%%%%%%%%%%%%%%%%%%%%%%%%%%%%%%%%%%%%%%%%%%%%%%%%%%%%%%%